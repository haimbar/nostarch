\documentclass[12pt, titlepage]{article}
\usepackage{enumerate, titlesec, natbib, url,pdfpages}
\usepackage[margin=1in]{geometry}

\usepackage[colorlinks=true, citecolor=blue]{hyperref}

\newcommand{\blue}[1]{{\textcolor{blue}{#1}}}
\newcommand{\red}[1]{{\textcolor{red}{#1}}}
\newcommand{\brown}[1]{{\textcolor{brown}{#1}}}
\newcommand{\green}[1]{{\textcolor{green}{#1}}}
\newcommand{\jy}[1]{{\textcolor{red}{JY: #1}}}


\titleformat{\subsection}{\bf}{Chapter \thesubsection}{1em}{}
\renewcommand{\thesubsection}{\arabic{subsection}}

\def\R{\textsf{R} }

\begin{document}
\author{Haim~Bar, HaiYing~Wang, and Jun~Yan}
\title{Book Proposal\\[5mm]
  \bf{\huge Stumbling into Data Science}\\
  A Simulation-Based Introduction}
\date{}
\maketitle


\tableofcontents

\pagebreak

\section{Description}
Data science is quite possibly the fastest growing research and business area today. It
requires skills and knowledge in statistics and computer science. Data scientists must
also have excellent communications skills in order to convey often-complex ideas in
a simple, truthful, and effective way to an audience which may not have quantitative
training. The demand for data scientists is expected to continue increasing, as our 
society becomes more and more data driven, and as the amount and variety of data
grows at an exponential rate.


There are many books on data science. Many of them focus on the computer-science side
of things, such as database maintenance, or data acquisition. Other books focus on the
statistical theories such as regression methods, or machine learning methods. We believe that
both aspects (computational and statistical) have to be combined, and every data scientist must
become (eventually) proficient in both.


Of books on the statistical aspect of data science, the overwhelming majority
rely on a certain level of mathematical background. This tends to intimidate
many students who shy away from anything with a strong
mathematical flavor. Even for students with a mathematical aptitude,
most books, even modern ones, are written in a traditional way which does
not illustrate the logical connection between the fundamental data science
procedures and the underlying mathematical justification. The advantage of
computer programming has not been taken in engaging future data scientists.


Our book aims to introduce some of the most fundamental ideas in statistical
data science to a general audience who are interested in data science.
Instead of equations and formulas, we will explain the same ideas
via computer simulations to engage the readers at the first sight.
The content will be accessible to students
with basic knowledge in math (middle-school algebra). The book will contain
complete and well documented code, so that students will learn statistical and
computational skills simultaneously. The code will be in \R, but the ideas and the simulation-based
approach is easily extendible to other languages, such as Python or Julia. The code and data
will be publicly available in a GitHub repository which will be constantly maintained by the authors.


The book can be used as a textbook in high-school classes, as well as in college classes for students
who have no background in statistics or computing and have no advanced mathematical training. It can
also be used as a self-study guide, for example by executives who have to deal with data or
manage data scientists, but lack the statistical or computational training. 

\section{About the Authors}

\textbf{Haim Bar} is an Associate Professor in the Department of Statistics,
University of Connecticut. Before joining UConn, he worked as a software
engineer for Motorola, a director of software development in MicroPatent, LLC, a
Principal Scientist at ATC-NY, and a statistician at Cornell's Statistical
Consulting Unit.
His research interests include Bayesian methods, statistical modeling and variable 
selection in high-dimensional data, especially in applications to genomics.


\noindent\textbf{HaiYing Wang} is an Associate Professor in the Department of
Statistics, University of Connecticut. He was an Assistant Professor in the
Department of Mathematics and Statistics at the University of New Hampshire and
an cooling system engineer in the Midea Group. His research interests include
informative subdata selection for big data, model selection, model averaging,
measurement error models, optimal design, and semi-parametric regression.

\noindent\textbf{Jun Yan} is a Professor in the Department of Statistics at the
University of Connecticut. a Research Fellow in the Center for Population Health
at UConn Health. He received his PhD in Statistics from University of
Wisconsin--Madison in 2003. After four years on the faculty of the Department of
Statistics and Actuarial Science at the University of Iowa, he joined
UConn in 2007. His methodological research interests include
survival analysis, clustered data analysis, spatial extremes, and statistical
computing. His application domains are public health, environmental sciences,
and sports. With a special interest in making his statistical methods available
via open source software, he and his coauthors developed and maintain a
collection of \R packages in the public domain. He
is an Elected Member of the International Statistical Institute and a Fellow of
the American Statistical Association.


All three authors have been members of the Computer Committee of the Department
for years, during which time the idea of the book emerged.

\section{Table of Contents}

\begin{enumerate}[{Chapter} 1 -]
\item Getting to Know R
\item Summarizing and Visualizing Data
\item Probability and Paradoxes
\item Towards the End of the Horizon
\item Estimation: A Hide-and-Seek Game with the Nature
\item Data Collection and Selection Bias
\item Correlation and Regression
\item Hypothesis Testing
\item Case Studies with New York Open Data
\end{enumerate}

\section{Annotated Table of Contents}

\subsection{Getting to Know R}
Just like a chef needs a set of tools, a kitchen with a large surface to work
on, and a detailed cookbook with different recipes, a data scientist needs a
powerful programming language, a convenient development environment, and good
documentation.


We chose the \R language and the RStudio integrated development environment
(IDE) for this book, and we hope that this book will be your basic guide into
data science. In order to get started, you must first get the necessary software
and get acquainted with it, and this is what this chapter is about.


\subsection{Summarizing and Visualizing Data}
People are born with a fantastic ability to detect patterns. However, sometimes the
information is hidden and the only way to observe a pattern is to change 
the viewpoint or rearrange the input. In this chapter we will introduce some tools
to summarize data in order to reveal hidden features. We will also discuss principles of
graphical excellence. In the words of Edward Tufte, the guru of data visualization,
`\textit{Graphical excellence consists of complex idea communicated with clarity, precision,
and efficiency.}'.


\subsection{Don't Trust Your Intuition: Probability and Paradoxes}
The world is random and probability is a tool to quantify the randomness. With
random events, our intuition may give us high confidence on a wrong
conclusion. For example, knowing that A is larger than B and B is larger than C,
can we say A is always larger than C? Not really for random events.

This chapter presents examples to demonstrate how probability helps
interpret `unusual' phenomena in real life and how may fool us. Simulation is a
great technique to help us understand these phenomena and find the counter-intuitive answers.


\subsection{Towards the End of the Horizon}
Benjamin Franklin once said `In this world nothing can be said to be certain,
except death and taxes'. Still, for all the uncertain things in life, and
especially in science where most theories cannot be proven correct, we would
like to be able to say something about how confident we are about a conclusion
we draw. To do that, scientists conduct an experiment in which they collect data
and check if it provides evidence in favor or against a proposed theory.


This chapter describes two of the most important results in probability and
statistics --- the law of large numbers and the central limit theorem. These
theorems explain why statistics works and is useful in so many situations! They
tell us that, if we collect enough data (and conduct a sensible experiment) we
will be able to say to what extent we can trust our results, and also tell us
how much data we need in order to draw conclusions with a certain degree of
confidence. Without these results, conducting experiments would be meaningless,
no matter how many samples are collected.

\subsection{Estimation: A Hide-and-Seek Game with the Nature}

Estimation is a fundamental question in data science applications. Often times,
one needs to estimate some target quantity of a population based on a
sample. How can we say one estimator is better than another? We would expect a
good estimator to hit the target on average with as small as possible uncertainty.


The estimation problem can be thought of as a hide-and-seek game that we play
with the Nature. The target is hidden by the Nature. Each estimator is a
strategy to find it with some confidence level. The hide-and-seek game can be
played easily with computer simulations, through which we can appreciate the
difference across the estimators.

\subsection{Data Collection and Selection Bias}
You can't make bricks without straw! Just like a cook may not make healthy and
tasty dishes from rotten food, a data scientist may not make valid conclusion
and estimation without appropriate data.

This chapter discusses the importance of obtaining unbiased data and
representative sample. Common sources of biases will be identified and
illustrated with real life examples. Some basic sampling designs for
representative samples will be introduced. 


\subsection{Correlation and Regression}
Hydrostatic weighing is one of the most accurate ways to measure body fat, but the procedure is inconvenient and unpleasant as it requires to
submerge people in a tub of water to weigh them. This can be avoided by using a
statistical model which uses a number of convenient body measurements to
estimate body fat. 

This chapter focuses on modeling the relationship between one variable of
interest to other variables. Linear regression and least squares estimation will
be discussed. Logistic regression which is commonly used to model binary data and for classification
will also be introduced. 

\subsection{Hypothesis Testing}

Hypothesis testing is a fundamental tool in scientific research. A hypothesis
can be either true or false. It can be either rejected or accepted based on
whether it is supported by the observed data.  Our decision can be either
correct or wrong. A good test procedure should control the probability of
falsely rejecting a true hypothesis, also known as type-I error, while maximizing
the probability of correctly rejecting a false hypothesis (maximize tthe `power').

Through computer simulations, where the truths are known, different test
procedures for the same hypothesis are compared in their type-I error rates and
power in this chapter. We expect a good test to maintain a low type-I error rate
and detect any deviation from the null hypothesis as much as possible. All tests
should high power as the sample size increases.

\subsection{Case Studies with NYC Open Data}

\href{https://opendata.cityofnewyork.us}{NYC open data} are free public data
published by NYC agencies and other partners. One of the most popular datasets is
the \href{https://data.cityofnewyork.us/Social-Services/311-Service-Requests-from-2010-to-Present/erm2-nwe9}{3-1-1 service requests from 2010 to present},
which contains 41 variables about each service request, including their creation
date, closed date, agency, complain type, and even latitude and longitude of the
location. We will use subsets of this dataset to illustrate some of the concepts
introduced in the earlier chapters. Our examples will help readers to jumpstart
their own analysis and visualization that may lead to novel, interesting and impactful
observations an conclusions.

\section{Sample Chapter}

Chapter~3: Paradoxes and Probability (attached).


\section{The Competition}
\begin{itemize}
\item Practical Data Science with R, by Nina Zumel and John Mount (2014).   \href{https://www.manning.com/books/practical-data-science-with-r}{ISBN: 9781617291562}.
\item Computer Age Statistical Inference:
Algorithms, Evidence and Data Science, by Bradley Efron and Trevor Hastie (2016).  \href{https://hastie.su.domains/CASI/}{ISBN-13: 978-1107149892}.
\item \R for Data Science, by Hadley Wickham and Garrett Grolemund (2017)  \href{https://r4ds.had.co.nz/}{ISBN-13: 978-1491910399}.
\item \R in Action, by Robert I, Kabacoff (2018). \href{https://rkabacoff.com/publication/r-in-action/}{ISBN-13: 978-1935182399}.
\item Data Science from Scratch: First Principles with Python, by Joel Grus
  (2019, 2nd
  Edition). \href{https://www.oreilly.com/library/view/data-science-from/9781492041122/}{ISBN:
    9781492041139}. 
\item Modern Data Science with R, by Benjamin S. Baumer, Daniel T. Kaplan,
  Nicholas, J. Horton (2021). \href{https://www.routledge.com/Modern-Data-Science-with-R/Baumer-Kaplan-Horton/p/book/9780367191498}{ISBN: 9780367191498}.
% \item The ABCs of Data Science: By Real Data Scientists for Future Data
%   Scientists (Very Young Professionals), by Rikin Mathur
%   (2020). \href{https://www.theabcsofdatascience.com}{ISBN: 978-1734276305}.
\item Data Science for Dummies, by Lillian Pierson (2021, 3rd
  Edition). \href{https://www.wiley.com/en-gb/Data+Science+For+Dummies,+3rd+Edition-p-9781119811558}{ISBN:
9781119811558}. 
\item Data Science for Beginners, by Andrew Park
  (2021). \href{https://www.amazon.com/Data-Science-Beginners-Programming-Learning/dp/B088N2DKVH}{ISBN-13:
    979-8645845551}. 
\item Practical Statistics for Data Scientists, by Peter Bruce, Andrew Bruce,
  and Peter Gedeck (2020, 2nd
  Edition). \href{https://www.oreilly.com/library/view/practical-statistics-for/9781492072935/}{ISBN:
    9781492072942}.
\end{itemize}

None of these books target the same audience as ours. In addition, our book
focuses on using computation and simulation to introduce and illustrate core
statistical concepts and techniques. The  books listed above either
focus on the \R programming language itself or on the statistical
methods presented at a high level. Our book emphasizes leaning by doing and
readers will learn by implementing and playing with the examples provided in
each chapter, following Confucius's philosophy on ``Tell me and I will forget,
show me and I may remember; involve me and I will understand.''


\section{Contact Information}

Haim Bar, Ph.D.\\
Department of Statistics\\
University of Connecticut\\
Email: haim.bar@uconn.edu

\newpage

\includepdf[pages={39-50}]{../main.pdf}

% \bibliographystyle{chicago}
% \bibliography{../book.bib,../packages.bib}

\end{document}

Xunzi Quote:
不闻不若闻之,闻之不若见之,见之不若知之,知之不若行之。学至于行之而止矣。
