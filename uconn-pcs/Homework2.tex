\documentclass{article}
\usepackage[hmargin={0.5in, 0.5in}, vmargin={0.8in, 0.8in}]{geometry}

\begin{document}

\begin{center}
\Large{Homework 2}\\
Data Science, 
PCS – UConn, 2022\\
Due Date: July 13, 2022 by 8pm
\end{center}

\begin{enumerate}
\item On HuskyCT you will find a file with the season's best results for men’s long-jump, since 1960 (longjump1.txt). Data were retrieved from Wikipedia: \verb|https://en.wikipedia.org/wiki/Long_jump#Men_5|
Use the following command to read the data:
\begin{verbatim}
longjump <- read.csv("longjump1.txt",sep="\t",header=TRUE)
\end{verbatim}
If you get an error that the file is not found, make sure you set the working directory.
\begin{enumerate}
\item	Use the commands you learned in class to create a frequency table, by the top athlete’s nationality.
\item	Use the commands you learned in class to find the minimum, maximum, and mean result across all years.
\item	Create a plot with the year on the x-axis and the best result on the y-axis. Create an aesthetically pleasing plot with the options you saw in the notes. Comment on the plot and describe the most striking observations which can be seen in the plot.
\end{enumerate}

\item	Run the following code to generate a data frame called ucb:
\begin{verbatim}
gender <- rep(c("female","male"),c(1835,2691))
admitted <- rep(c("yes","no","yes","no"),c(557,1278,1198,1493))
dept <- rep(c("A","B","C","D","E","F","A","B","C","D","E","F"),
           c(89,17,202,131,94,24,19,8,391,244,299,317))
dept2 <- rep(c("A","B","C","D","E","F","A","B","C","D","E","F"),
            c(512,353,120,138,53,22,313,207,205,279,138,351))
department = c(dept,dept2)
ucb <- data.frame(gender,admitted,department)
rm(gender,admitted,dept,dept2,department)
ls()
\end{verbatim}
\begin{enumerate}
\item	Use one of the functions you saw in class and the notes to create a summary of each variable.
\item	Create a contingency table of gender by department called GenderDept, and include the row and column sums.
\item	Using the table you created, plot a spinogram.
\item	Dose there seem to be a relationship between department and gender? 
\end{enumerate}


\item	“airquality” is a built-in data set in R. It has 154 observations and 6 variables. Read the description by yourself by typing ?airquality
\begin{enumerate}
\item	Get summary statistics on all 6 variables in “airquality”.
\item	Draw a boxplot of the Temp variable.
\item	Plot the kernel density graph of Wind by each month. And compare them on one graph. Add necessary titles, and legend.
\item	Are there any conclusions you can draw from the plots and tables? Feel free to analyze additional variable.
\end{enumerate}
\end{enumerate}

\end{document}

