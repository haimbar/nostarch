\chapter{A Case Study: NYC Crashes}

Cities never sleep, and neither does their traffic. Every day thousands
of people move through the streets by car, bus, bike, and on foot, and
occasionally things go wrong. When a crash happens in New York City, the
details are recorded by the police and stored in a large public
database. These records give us a chance to see how data describe real
events in a busy city, but they also come with quirks, gaps, and
surprises. In this chapter we use crash data around Labor Day to see how
open data can help us understand the world, as long as we look carefully
and check our assumptions.


The crash dataset we use contains 29 variables, including the date and
time of the crash, location information, borough, types of vehicles
involved, and numbers of injuries and fatalities. We begin by reading
the data, cleaning the variable names, and checking whether the entries
actually fall within the week we intend to study. Then we examine
missing values, create new variables such as the hour of the crash and
whether the day is a business day, and explore temporal and spatial
patterns. Finally, we study severity, contributing factors, and vehicle
types. The goal is to follow the full analysis workflow in a transparent
and reprosducible way.

\section{Importing and Inspecting the Data}

We first load the dataset and take a quick look at its structure.

\showChunk{R}{Code/nyc_crashes.R}{read}

The variable names are all upper case with spaces or periods. To make
them easier to work with, we convert them to lower case and replace
spaces and periods with underscores.

\showChunk{R}{Code/nyc_crashes.R}{rename}

With cleaner variable names, we can begin to investigate the range of
dates in the dataset.

\showChunk{R}{Code/nyc_crashes.R}{time_window}

The raw data span eight consecutive days. This is longer than a typical
seven-day week, and it is a good reminder that we should not take
anything for granted. Sometimes a “week” begins on Monday, other times
on Sunday. The raw file includes the Sunday before Labor Day as well as
the following Sunday. We now filter the dataset so that it covers the
Sunday--Saturday interval that we intend to study.

\showChunk{R}{Code/nyc_crashes.R}{filter_week}

The \texttt{location} variable in the raw data stores a text version of
the coordinates. Because latitude and longitude are already available,
the \texttt{location} field is redundant and sometimes inconsistent with
the numeric values. The following code inspects this variable.

\showChunk{R}{Code/nyc_crashes.R}{location_check}

\section{Missing Values and Data Quality}

Several fields have missing entries. We begin with the borough and zip
code variables. Some borough fields are empty strings rather than
missing values, so we convert them to \texttt{NA}. We also check cases
where both borough and zip code are missing.

\showChunk{R}{Code/nyc_crashes.R}{missing_borough_zip}

Location information has its own set of issues. A small number of
crashes report coordinates at \texttt{(0,0)}, which is not a location in
New York City. These should be treated as missing values.

\showChunk{R}{Code/nyc_crashes.R}{missing_geo}

When a crash has valid latitude and longitude but is missing a zip code,
it is in principle possible to fill in the missing zip based on its
location. We define a logical variable \texttt{fillable\_zip} that
indicates such cases. We then compare the proportion of fillable zip
codes across days of the week.

\showChunk{R}{Code/nyc_crashes.R}{fillable_zip}

\section{Exploring Temporal Patterns}

Next we create a \texttt{datetime} variable and extract the hour and
minute of each crash. This allows us to examine how crashes vary across
the day and across boroughs.

\showChunk{R}{Code/nyc_crashes.R}{hour}

There is also a striking number of crashes recorded exactly at midnight
or exactly on the hour. This pattern may reflect reporting conventions
rather than the actual times of the crashes.

To compare business days and non-business days, we define a
\texttt{business\_day} variable and inspect the number of crashes in each
borough for each type of day.

\showChunk{R}{Code/nyc_crashes.R}{business_day}

\section{Crash Severity}

A crash is considered severe if at least one person is injured or
killed. We construct a logical variable \texttt{severe} for this purpose.

\showChunk{R}{Code/nyc_crashes.R}{severity}

Crashes can involve different numbers of motor vehicles. We count the
number of non-missing vehicle type fields to form the variable
\texttt{n\_vehicles}, and examine how severity relates to the number of
vehicles involved.

\showChunk{R}{Code/nyc_crashes.R}{vehcount}

Severity also varies across the hours of the day. The following
contingency table examines this relationship.

\showChunk{R}{Code/nyc_crashes.R}{severe_by_hour}

The most severe crashes, as measured by injuries and fatalities, can
also be identified and inspected individually. Several of them occur in
clusters within particular neighborhoods.

\showChunk{R}{Code/nyc_crashes.R}{top_severe}

\section{Contributing Factors and Vehicle Types}

Each crash record contains up to five contributing factors. We begin by
studying the factor listed for vehicle~1. Because the entries differ
only in capitalization or include vague values such as ``unspecified,''
we standardize them and treat certain values as missing.

\showChunk{R}{Code/nyc_crashes.R}{contrib_factors}

Finally, we look at the types of vehicles involved in the crashes. The
most common categories reflect everyday traffic in New York City, such
as sedans and sport utility vehicles.

\showChunk{R}{Code/nyc_crashes.R}{veh_types}

\section{Summary}

This case study showed how the ideas introduced throughout the book
naturally lead to practical investigation. By loading the data, checking
its structure, cleaning fields, creating new variables, and exploring
patterns, we followed the same steps you can use on almost any dataset.
Nothing here required new tools—only the habits of curiosity,
skepticism, and careful reasoning that you have been building. New York
City’s Open Data portal is only one place to find material for your own
projects; many cities, federal agencies, and scientific groups share
their data freely as well. Transportation records, environmental
measurements, public health reports, sports statistics, and countless
other datasets are available to explore. With the tools you have
stumbled into, you can now begin asking your own questions and making
your own discoveries. This chapter is one example of what is possible.
The next ones are yours to find.




%%% Local Variables:
%%% mode: latex
%%% TeX-command-extra-options: "-shell-escape"
%%% TeX-engine: xetex
%%% TeX-master: "../sidsmain.tex"
%%% End:
