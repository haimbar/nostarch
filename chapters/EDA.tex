\chapter{Summarizing and Visualizing Data}\label{EDA}
\chapterartfile{images/static/Rlogo.png}
\hypertarget{EDA}{%
People are born with a fantastic ability to detect patterns. However, sometimes the
information is hidden and the only way to observe a pattern is to change
the viewpoint or rearrange the input. In this chapter we'll introduce some tools
to summarize data in order to reveal hidden features. We'll also discuss principles of
graphical excellence. In the words of Edward Tufte, the guru of data visualization,
`\textit{Graphical excellence consists of complex idea communicated with clarity, precision,
and efficiency.}'
}

%https://python-graph-gallery.com/barplot/

% \runR{Code/ch2sec1.R}{ch2sec1}%[cache]

\hypertarget{TypesOfData}{%
\section{Types of Data}}
In this book we cover several methods to analyze the data with statistical
tools. However, it's very important to always look at the data, and this means
creating plots and tables in order to understand it better. We call this process
``exploratory data analysis'', or EDA, and it's a critical step which should
never be skipped. The type of plot which we generate in each case depends on the
data, and usually there's more than one option to choose from. In this chapter
we cover several visualization methods, and in subsequent chapters we will see
additional ones.

It's useful to think of data as a table in which each row represents one
individual, and each column corresponds to some characteristic of
individuals. These characteristics are often called `variables', because they
vary across individuals. There are three broad types of variables:
\begin{itemize}
\item \textit{Quantitative}: a variable that can take numerical values. It
  includes two sub-types.
  \begin{itemize}
  \item \textit{Continuous}: a variable that can take any value within
    some range. For example, a person's income can take any value greater than
    zero. Sometimes the range of values is unbounded (from minus infinity to
    infinity), and sometimes it is bounded (like age, height, weight).
  \item \textit{Discrete}: a variable that can take at most a countable number
    of possible values. For example, the number of students in a class. 
  \end{itemize}
\item \textit{Categorical}: an individual's characteristic that can only take a finite number of values. For example, sex, marital status, or number of years of education are categorical. The values that a categorical value can take are called \emph{levels}.
\item \textit{Ordinal}: Categorical variables that have a natural order (like number of years of education).
\end{itemize}

In \R there are many other types of variables, but they are generally special cases of these three type of variables. Here are some important examples:

\begin{itemize}
\item \textit{Dates}: A date is stored in \R as an integer. Arbitrarily, January
  1, 1970 was set to 0, and the value for each other date is the number of days
  since 1/1/1970, accounting for leap years. Dates prior to that are given
  negative values.

\item \textit{Boolean}: Some variables are either \lstinline{TRUE} or
  \lstinline{FALSE}. In the coin-flipping simulation in Chapter~\ref{rintro} we
  generated a Boolean variable which was \lstinline{TRUE} if a randomly generated
  number drawn uniformly between zero and one was greater than 0.5, and
  \lstinline{FALSE} otherwise. Boolean variables are categorical, with two levels.

\item \textit{Character}: When a variable has a finite number of levels, \R
  allows to make a distinction between a `factor' type and a `character' type. A
  factor is used when the distinct levels of the variable are meaningful for
  analysis and plotting. For example, Sex, marital status, and education level
  are usually used as factors because we are interested in comparing between
  different levels. On the other hand, names and student IDs are usually of no
  interest and are only used as labels or identifiers for individuals in the
  data.

\item \textit{Vectors, matrices, and data frames}: Multiple values of the
  \textit{same type of variable} (e.g., numeric, factor, etc.) can be combined
  into a vector via the \lstinline{c()} function or generated via the \lstinline{rep()},
  \lstinline{seq()}, or random generating functions such as
  \lstinline{rnorm()}. Similarly, vectors of the \textit{same type} can be combined
  into a matrix -- a two dimensional rectangular arrangement of values. A data
  frame is a more general structure -- it is a two dimensional arrangement of
  values in which within each column it must contain the same type of variables,
  but different columns can contain different types of variables.
\end{itemize}

Run the following code:

\showCode{R}{Code/ch2-01gendat.R}
\runR{Code/ch2-01gendat.R}{ch2-01gendat}

Note that in the third line we create a variable called \lstinline{g} in two steps:
first, we draw a random sample of the same length as that of the vector \lstinline{x}
from a binomial distribution with probability 0.5. This gives a numeric vector
in which approximately half the values are 0 and half are 1. Then, we convert it
to a factor by using the function \lstinline{factor()}. Use the \lstinline{str()} function
to see the structure and levels of this variable. In line 4 we create the
variable we are interested in, \lstinline{y}, through some formula which we will
discuss later. In line 6 we create a Boolean variable, \lstinline{yL}, which is set
to TRUE if the variable \lstinline{y} exceeds the threshold (25) and FALSE
otherwise. We then combine the vectors \lstinline{x}, \lstinline{g}, \lstinline{y} and
\lstinline{yL} into one data frame called \lstinline{df1}.
Enter \lstinline{str(df1)} to see the structure of \lstinline{df1}. Note that the columns
in the data frame have different types.
The last line in the code above produces a summary of each column in the data
frame:
\inlnR{```print(summary(df1))```}[vbox]
% \includeOutput{ch2_01gendat}

For numerical variables like \lstinline{x} and \lstinline{y}, \R shows the mean and the
`five-number summary': the minimum, the maximum, and the 25th, 50th, and 75th
percentiles. For categorical variables such as \lstinline{g} and \lstinline{yL} it shows
the counts in each category. For \lstinline{yL} it also shows that it is a special
kind of categorical variable, namely, it is logical (Boolean).

In the following we are interested in understanding the characteristics of
\lstinline{y} and its possible relationship with other variables. In real life, we
never know the formula that generates a variable, so let us pretend that we
don't have access to lines 4-5 in the code above, for now.

%% lists?

%as.numeric(as.Date("1970-01-01"))
%as.numeric(Sys.Date())

\hypertarget{Plots}{%
\section{Plots}}

\subsection{Boxplot}
Using the data we generated in the previous section, we start by plotting
\lstinline{y}. The simplest graphical representation is the boxplot which was
introduced in Chapter~\ref{rintro}, and is constructed from the `five-number
summary'.
\showCode{R}{Code/ch2-02boxplot.R}[2][3]
\runR{Code/ch2-02boxplot.R}{ch2-02boxplot}

We can see the boxplot in Figure~\ref{boxploty}.
Note the usage of the options of the \lstinline{boxplot} function, which are
optional, but improve the way the plot looks. The \lstinline{axes=F} option means
that the x and y axes are not shown. In the next line we explicitly draw the
vertical axis.
Try changing the values in these options. To see which colors are available, you
can run the \lstinline{colours()} function.

\begin{figure}
\begin{center}
\includegraphics[width=0.6\textwidth]{images/chapter_2/boxploty.pdf}
\caption{A boxplot.}
\label{boxploty}
\end{center}
\end{figure}

What can we learn from this boxplot? Besides the five-number summary which is
depicted graphically, we can see that the distribution of \lstinline{y} is
approximately symmetric, and there are no outliers. However, we can't say much
more about how \lstinline{y} is distributed.

\subsection{Histogram}
Another way to look at the distribution of a single numeric variables is to draw
a histogram, as we did in Chapter \ref{rintro}. A histogram is constructed (by
\R) by dividing the range of a variable into `bins', and simply count the number
of observations in each bin. This is done with the \lstinline{hist()} function:
\showCode{R}{Code/ch2-03hist.R}[2][2]
\runR{Code/ch2-03hist.R}{ch2-03hist}

Again, notice the usage of the function's arguments. For example, \lstinline{breaks}
determines the number of bins, which \lstinline{col} and \lstinline{border} determine the
color of the interior and edges of each bin. The \lstinline{main=""} option is used
in order to remove the title of the plot. The result is shown in
Figure~\ref{histy}.
Try changing these values. What happens when you set \lstinline{breaks} to be
smaller, or larger?

\begin{figure}
\begin{center}
\includegraphics[width=0.6\textwidth]{images/chapter_2/histy.pdf}
\caption{A histogram.}
\label{histy}
\end{center}
\end{figure}

The histogram gives us more information than the boxplot. It shows a more
detailed picture of the distribution and, in this particular case we see that
although the distribution is indeed approximately symmetric, it is also bi-modal
- it has two noticeable peaks which we cannot see with a boxplot.

Is this enough? Can we use this plot to come up with a plausible explanation for
the particular shape of the distribution? If we have more variables we can try
to see if our variable \lstinline{y} is somehow related to any of the other
variables, which is what we will do next.

\subsection{Side by side boxplots}
In addition to \lstinline{y}, we have a categorical variable called \lstinline{g}. To see
if it is related to \lstinline{y} we can show one boxplot for each level of the
factor \lstinline{g}, which, in this case means two side-by-side boxplots. The syntax
to do this is as follows:

\showCode{R}{Code/ch2-04sidebox.R}[2][5]
\runR{Code/ch2-04sidebox.R}{04sidebox}

The notation \lstinline{y} $\sim$ \lstinline{g} 
means `show how \lstinline{y} varies by \lstinline{g}'. Again, we suppress the axes and
show only the vertical one. We also suppress the label of the x-axis using
\lstinline{xlab=""}. For each level of the factor \lstinline{g} we use a different
combination of border and interior colors. We set the vertical range of the plot
using \lstinline{ylim=c(min(y), max(y)*1.1))}, so that it will have some extra space
at the top of the figure. We use this space to put the labels for each group
with the \lstinline{text()} function.

The first argument of the \lstinline{text()} function contains the x-coordinates of
the text. In this case we have two values: 1 and 2. Why not 0 and 1, which are
the actual levels of the factor \lstinline{g}? This is because, within \R levels of
factors are stored numerically from 1 to the number of levels of the factor. The
ordered is determined alphabetically by default, so 0 is represented as 1, and 1
as 2.

The second argument contains the y-coordinates of the text. We set the maximum
value on the vertical axis to be max(y)*1.1, so we put the text between the
maximum value of \lstinline{y} and the upper limit of the vertical axis, so that the
text does not cover the boxplot, nor is it outside the range of the plot.

The third argument is the actual text we want to display, and it is followed by
\lstinline{cex=0.6} which sets the font size to be 60\% of the default.

\begin{figure}
\begin{center}
\includegraphics[width=0.6\textwidth]{images/chapter_2/sidebysidebp.pdf}
\caption{Side by side boxplots.}
\label{sidebysidebp}
\end{center}
\end{figure}

Is there a difference between the two groups, as determined by \lstinline{g}? It's
not entirely clear. The medians are about the same but the box of group 2 is
slightly elevated.

We can also plot a histogram for each group. We can do it in two plots, or in
one, which makes the comparison between the groups easier. To do that we use the
following code. The result is depicted in Figure~\ref{histoverlay}.
\showCode{R}{Code/ch2-05histoverlay.R}[2][6]
\runR{Code/ch2-05histoverlay.R}{05histoverlay}

When we use the brackets, as in \lstinline{y[g==0]}, we tell \R to apply the
histogram function only to the subset of the data for which \lstinline{g} has the
value 1.

We use the \lstinline{xlim} argument to make sure that the range of the horizontal
axis will be sufficient in order to show all the data. We need to do that here,
because our first plot uses only half the data and \R uses it to infer what the
range should be, and it chooses the range based on the subset, not the entire
data.

The second time we call \lstinline{hist} we use the \lstinline{add=TRUE} option, which
means that the second plot will be drawn on top of the first one. In order for
the second histogram not to hide the first one, we set the color of the second
histogram to be partially transparent. To do that, we have to specify the color
in RGB (Red, Green, Blue) format. Every color can be constructed as a mixture of
these three base colors. Computers use 256 levels (from 0 to 255) for each of
the base colors, and they represent them as hexadecimal numbers (base 16,
instead of the familiar decimal, or base 10 which we use in our daily lives.) It
may be slightly awkward to think in hexadecimal but, in order to get the code
for the table you need there are tables to convert from a color name to a
number. \R also has function to help you do that. Suppose we want to use the
color "navyblue". If we run \lstinline{col2rgb("navyblue")} we get the numbers 0,0,
and 128, which means that the mixture has 0 red, 0 green, and the intensity for
the blue is 128/255. We then run \lstinline{rgb(t(col2rgb("navyblue"))/255)} and get
\#000080. Each pair of digits corresponds to the three base colors. Then, to add
transparency we can do \lstinline{rgb(t(col2rgb("navyblue"))/255, alpha = 0.5)} add
another hexadecimal number (with alpha=0.5 we get 80.) For complete transparency
we use alpha=0, and for no transparency we use alpha=1.

Doing this helps to see that there really is a difference between the groups,
and the blue histogram is similar in shape to the one in orange, but more
shifted to the right. The bi-modality is common to both groups.

\begin{figure}
\begin{center}
\includegraphics[width=0.6\textwidth]{images/chapter_2/histoverlay.pdf}
\caption{Overlaid histograms.}
\label{histoverlay}
\end{center}
\end{figure}


\subsection{Scatterplot}
Next, we want to see if the variable \lstinline{y} is related to \lstinline{x} in some
way. Since both are continuous variables, we can display the relationship
between them with a scatterplot. In the following code we use the \lstinline{plot()}
function, which \R infers to mean a scatterplot of the first argument \lstinline{x}
vs. the second, \lstinline{y}, because both are continuous. The other arguments to
the function should be familiar by now.
\showCode{R}{Code/ch2-06scatter.R}[2][4]
\runR{Code/ch2-06scatter.R}{ch2-06scatter}

In the left panel of Figure~\ref{scatter} we can see an interesting pattern
which we could not observe in the previous plots. First, there seems to be a
cyclic pattern. Second, in the first half of the x-range, \lstinline{y} appears to
remain constant (except for the cyclic pattern). However, in the second half we
see that there is a distinct `jump' which is followed by an increasing trend.
We also see that the data is quite noisy. If we draw (in our mind's eye, for
now) a line that represents the relationship between \lstinline{x} and \lstinline{y}, the
points will be scattered around that line.

\begin{figure}
\begin{center}
\includegraphics[width=0.48\textwidth]{images/chapter_2/scatter.pdf}
\includegraphics[width=0.48\textwidth]{images/chapter_2/scatter2.pdf}
\caption{Scatterplots.}
\label{scatter}
\end{center}
\end{figure}

\subsection{Adding dimensions to a plot}
So far we saw that there is a complicated relationship between \lstinline{x} and
\lstinline{y}, and maybe some relationship between \lstinline{g} and \lstinline{y}. Can we put
all three variables in one plot, and maybe uncover more complex patterns? Yes!
We can use different colors, shapes, or sizes for each point. For example, we
can use different colors for each group, as follows:
\showCode{R}{Code/ch2-07scatter2.R}[2][6]
\runR{Code/ch2-07scatter2.R}{ch2-07scatter2}

First, we use the \lstinline{ifelse()} function to set the color for each group. The
first argument is the binary rule we use, the second is the value we assign when
the first argument is TRUE, and the second is the value being assigned when the
first argument results in a FALSE value. In this case, all the points for which
\lstinline{g} is equal to 1, are drawn in navy blue, and all the others in orange.

% \begin{figure}
% \begin{center}
% \includegraphics[width=0.6\textwidth]{images/chapter_2/scatter2.pdf}
% \caption{A scatterplot with group colors}
% \label{scatter2}
% \end{center}
% \end{figure}

The right panel of Figure~\ref{scatter} shows the result. Now, we can see the
entire picture. The difference between the groups seems real, and both have a
cyclic pattern and a distinct change-point where the value of \lstinline{y} increases
abruptly, and then continues to increase as \lstinline{x} becomes larger.

We also added a horizontal reference line at \lstinline{y=30} by using the
\lstinline{abline()} function. We set its type to dashed, its color to red, and its
width to 3 times the default width, in order to create a clear visual effect. It
is also possible to add vertical or slanted reference lines using the same
function.

We can imagine a hypothetical situation where \lstinline{y} is a temperature of an
air-conditioned storage facility, with cyclic fluctuations which perhaps
correspond to the time of day. Then, at some point the A/C system malfunctioned,
and the temperature increased and then continued an upward trend as heat was
building up. An alarm was set off as soon as the temperature exceeded the
threshold of 30.


%y <- 3*(x>2*pi)*x + 4*sin(3*x) + 10*(as.numeric(g)-1) + rnorm(length(x), mean = 0, sd = 3)

\begin{exercise}
Try running the simulation code, but with the following changes:
\begin{enumerate}
\item Without the $x$ term: \lstinline{y <- 10 * (as.numeric(g) - 1) + rnorm(length(x), mean=0, sd=3)} 
\item Without the cyclical term: \lstinline{y <- 3 * (x > 2 * pi) * x +  10 * (as.numeric(g) - 1) + rnorm(length(x), mean=0, sd=3)} 
\item Without the group term: \lstinline{y <- 3 * (x > 2 * pi) * x +  rnorm(length(x), mean=0, sd=3)} 
\item Without the random noise: \lstinline{y <- 3 * (x > 2 * pi) * x + 4 * sin(3 * x) + 10 * (as.numeric(g) - 1)} 
\item With less noise: \lstinline{y <- 3 * (x > 2 * pi) * x + 4 * sin(3 * x) + 10 * (as.numeric(g) - 1) + rnorm(length(x), mean=0, sd=1)} 
\item With more noise: \lstinline{y <- 3 * (x > 2 * pi) * x + 4 * sin(3 * x) + 10 *( as.numeric(g) - 1) + rnorm(length(x), mean=0, sd=5)} 
\item Change the coefficients. Include negative values in your attempts.
\item Use a different marker in the plot (using the \lstinline{pch} option).
\end{enumerate}
\end{exercise}

\hypertarget{tables}{%
\section{Tabulating and cross-tabulating data}}
In the previous section we were interested in a continuous variable, and how it
may depend on other variables, both continuous and categorical. When we want to
explore and investigate categorical variables we often use tables, or their
graphical representation.

Let's simulate such data:
\showCode{R}{Code/ch2-08gendat2.R}
\runR{Code/ch2-08gendat2.R}{ch2-08gendat2}

We use the \lstinline{rpois()} function to draw random numbers which are all
non-negative integers (from the Poisson distribution). Use what you learned in
the previous section to plot this data. We then create categories by using the
\lstinline{cut()} function is used to create categories from continuous data. It can
then be summarized by using the \lstinline{table()} function, which simply counts the
total number of observations in each category:
\inlnR{```print(table(xcat))```}[vbox]
This can be shown graphically by using the \lstinline{barplot()} function as shown in
Figure~\ref{barplot} (or as a pie chart by using the \lstinline{pie()} function.)
\showCode{R}{Code/ch2-09barplot.R}[2][3]
\runR{Code/ch2-09barplot.R}{ch2-09barplot}

\begin{figure}
\begin{center}
\includegraphics[width=0.8\textwidth]{images/chapter_2/barplot.pdf}
\caption{A barplot.}
\label{barplot}
\end{center}
\end{figure}

When there is more than one variable we can use cross-tabulation, which is also referred to as a `contingency table'. Let's create a new variable called \lstinline{ycat}:
\showCode{R}{Code/ch2-10genycat.R}
\runR{Code/ch2-10genycat.R}{ch2-10genycat}

The function \lstinline{ftable()} (for `flat table') is used to cross-tabulate.
\inlnR{```print(ftable(xcat, ycat))```}[vbox]

There are a few things which this table allows us to see quite easily. It is clear that the distribution in each row is different from the other ones. The same thing can be said about the columns. 
For example, most of the observations in the (0,30] category of the y variable are in (0,9] or (9,12] categories of the x variable. However, most of the data in the (100,300] y columns are in the (12,16] or (16,25] rows.

To visualize this table in a plot, we can create a `spine plot', with the \lstinline{spinogram()} function, which shows the table of proportions, rather than counts, which means that the bars all have the same total width and height, and the dimensions of boxes within each bar correspond to the proportion of the corresponding cell in the table relative to the total in the row/column of that cell. This is depicted in Figure~\ref{spineplot}. Equivalently, we can also summarize this data in the form of a table by using the function \lstinline{prop.table()}.
\showCode{R}{Code/ch2-11proptable.R}[2][3]
\runR{Code/ch2-11proptable.R}{ch2-11proptable}

This simulation can represent a hypothetical job market where the x categories
correspond to years of education, and the y categories to levels of income (in
\$10,000). The levels for \lstinline{x} correspond to 1. dropped out before
high-school, 2. attended or finished high school, 3. college student, and
4. people who studied for an advanced degree. The relationship between the
continuous-data variables may not be linear (in this case, we know it is
quadratic because we simulated it this way) and when there is no reason to think
that there is some simple formula to represent the relationship, or when all we
have is the categorical data, cross-tabulation is the appropriate way to
visualize and understand the data.

\begin{figure}
\begin{center}
\includegraphics[width=0.6\textwidth]{images/chapter_2/spineplot.pdf}
\caption{A spine plot.}
\label{spineplot}
\end{center}
\end{figure}

\hypertarget{beyond}{
\section{Going further}}
There is \textit{much} more to learn about effective visualization. There are
many plotting tools and packages which greatly enhance the number of
possibilities for data visualization. In \R, one can use the very popular and
feature-rich ggplot2 package \citep{ggplot2}, and other ones such as plotrix
\citep{plotrix}, gplots \citep{warnes2016package}, plotly \citep{plotly}, etc. 
You should also explore graphical tools in other languages can be used. For example, Matlab and Python use the very comprehensive and aesthetic matplotlib package, and for interactive plots you may find JavaScript to be very attractive with its many excellent graphing options.

Regardless of which tools you use, the important thing to keep in mind is that effective data visualization is critical and sometimes more persuasive than formal statistical analysis. Thus, we want to conclude this chapter with a few important takeaways.

\begin{itemize}
\item Good visualization is almost always multivariate: we learn a lot when we
  compare and contrast things. Side by side boxplots, scatterplots, and
  cross-tabulation are a few techniques to allow us to put things in
  perspective.

  % overlayed histogram, density
\item It is critical that the plot, table, or summary of the data is accurate
  and honest. Show the data and insights obtained from the visualization without
  introducing any bias. For example, if you exclude points or rescale the plot,
  you must have a good reason to do so, and make this decision known to the
  audience of your visualization.

\item Always strive to produce effective data visualization, so that the
  conclusions are conveyed clearly and with little mental effort on the part of
  the reader. This will often require a substantial mental effort on your
  part. Do not assume that a graph that you find clear and effective also works
  for other people. Try to show it to many people and improve your presentation
  by taking into account other peoples' suggestions.

\item Choose the right tool for the job, and make the data the center of
  attention -- not the bells and whistles that some software allows you to
  produce. Edward Tufte \textcolor{red}{CITE} advocated for the principal of
  maximizing the `information to ink ratio', and Albert Einstein said
  `everything should be made as simple as possible, but not simpler.'

\item Don't just rely on excellent tools, and don't be lazy -- explore ways to
  make the visualization better. For example, choosing the type of graph, the
  point sizes, shapes, or colors, choosing the best scale, can all make a big
  difference.

\item Finally, we live in an era of big data, and for that we sometimes need to
  choose plots based on computational constraints. To demonstrate it, consider
  the following code:
  \showCode{R}{Code/ch2-12smoth.R}[2][5]
  \runR{Code/ch2-12smoth.R}{ch2-12smoth}

  We generate 100,000 points \lstinline{(x,y)} and construct a scatterplot (in line 3, which
  we commented out). Try running this line. The large number of points may
  actually hide some of the features of the data. The shape of the function is
  clear, but the points are so dense that it is hard to see that most of the
  points actually lie close to the center of the curve. When we plot a smooth
  version of the scatterplot (line 4) we get a much nicer plot, which we show in
  Figure~\ref{smooth}. Furthermore, the smooth scatterplot saves us a lot of
  space. In this example with 100,000 points, when we save the the plots in pdf
  format, the simple scatterplot takes 44MB, while the smooth version is more
  than 2000 time smaller and takes only 18kB. In real life you will come across
  much larger datasets, and then the size of the plot (in bytes) will make a big
  difference. It may take a long time to download a plot on the Internet, and
  for the viewer (like Adobe) to show it on the screen, or to print it.
\end{itemize}
\begin{figure}
\begin{center}
\includegraphics[width=0.6\textwidth]{images/chapter_2/smooth.pdf}
\caption{A smooth scatterplot.}
\label{smooth}
\end{center}
\end{figure}

%%% Local Variables:
%%% mode: latex
%%% TeX-command-extra-options: "-shell-escape"
%%% TeX-engine: xetex
%%% TeX-master: "../sidsmain.tex"
%%% End: